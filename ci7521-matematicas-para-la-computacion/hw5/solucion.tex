% ********************************************** %
% *                                            * %
% * Autor : Claro Mosqueda Alejandro           * %
% *                                            * %
% * e-mail: alejandro.claro@gmail.com          * %
% *                                            * %
% * Teoria de algoritmos, USB                  * %
% *                                            * %
% * Tarea No  5                                * %
% *                                            * %
% ********************************************** %

\documentclass[letterpaper,11pt]{article}

\usepackage[spanish]{babel}
\usepackage[latin1]{inputenc}
\usepackage{pifont}
\usepackage{mathrsfs}
\usepackage{stmaryrd, manfnt,marvosym,yhmath}
\usepackage{amsmath, amsthm, amssymb}
%\usepackage{amsfonts}
%\usepackage{graphicx}
%\usepackage{concmath}

\headheight 0.2truecm \headsep 0.5truecm \topmargin -0.54truecm
\topskip 0pt \textheight 22.49truecm \footskip 0.75truecm
\oddsidemargin 0.46truecm \evensidemargin 0.46truecm
\marginparwidth 0pt \marginparsep 0pt \textwidth 16.59truecm
\parindent 2em
\fboxsep 1ex \fboxrule 0.15ex \setlength{\jot}{0.2truecm}
\flushbottom

% Use $$\Braket{\phi | {\partial^2}\over {\partial t^2} | \psi}$$

\def\bra#1{\mathinner{\langle{#1}|}}
\def\ket#1{\mathinner{|{#1}\rangle}}
\def\braket#1{\mathinner{\langle{#1}\rangle}}
\def\Bra#1{\left<#1\right|}
\def\Ket#1{\left|#1\right>}
{\catcode`\|=\active 
  \gdef\Braket#1{\left<\mathcode`\|"8000\let|\bravert {#1}\right>}}
\def\bravert{\egroup\,\vrule\,\bgroup}

\newcommand{\pderiv}[2]{\frac{\partial #1}{\partial #2}}      % derivada parcial
\newcommand{\deriv}[2]{\frac{\mathrm{d} #1}{\mathrm{d} #2}}      % derivada parcial
\newcommand{\comm}[2]{\left[\,#1\;,\;#2\,\right]}                 % [ , ]
%\newcommand{\qed}{\hfill \ensuremath{\Box}}

%\newtheorem{theorem}{Theorem}[section]
%\newtheorem{lemma}[theorem]{Lemma}
%\newtheorem{proposition}[theorem]{Proposition}
%\newtheorem{corollary}[theorem]{Corollary}

%\newenvironment{proof}[1][Proof]{\begin{trivlist}
%\item[\hskip \labelsep {\bfseries #1}]}{\end{trivlist}}
%\newenvironment{definition}[1][Definition]{\begin{trivlist}
%\item[\hskip \labelsep {\bfseries #1}]}{\end{trivlist}}
%\newenvironment{example}[1][Example]{\begin{trivlist}
%\item[\hskip \labelsep {\bfseries #1}]}{\end{trivlist}}
%\newenvironment{remark}[1][Remark]{\begin{trivlist}
%\item[\hskip \labelsep {\bfseries #1}]}{\end{trivlist}}

%%%%%%%%%%%%%%% Comienzo del documento %%%%%%%%%%%%%%

\begin{document}

  \setlength{\unitlength}{1truecm}\thicklines%

  \begin{flushleft}
    \bf {\small UNIVERSIDAD SIM�N BOLIVAR\hfill \today} \\
    \bf {\small CI-7521 - Matem�ticas para la computaci�n\hfill Lic. Alejandro Claro Mosqueda} \\
  \end{flushleft}

  \bigskip

	\begin{center}
	\noindent \Large \bf Tarea \#5
	\end{center}

  \bigskip

\noindent {\bf Problema \#1:} Evalu� $\sum_{0<k<n} \frac{1}{k(n - k)}$ de dos formas diferentes:

\begin{enumerate}
	\item [a)] Expandiendo en fracciones parciales.
	\item [b)] Tratando la suma como una convoluci�n y utilizando funciones generatrices.
\end{enumerate}
\smallskip

\noindent {\bf Soluci�n}\\
\smallskip

\noindent a) Primero se deben determinar las fracciones parciales.

\begin{eqnarray*}
  \frac{1}{k(n - k)} & = & \frac{a}{k} + \frac{b}{n - k} \\
                     & = & \frac{a(n-k) + b\,k}{k(n - k)} \\
                     & = & \frac{(b-a)k + a\,n}{k(n - k)}
\end{eqnarray*}

\noindent Para que se cumpla la igualdad $(b-a)k + a\,n = 1$. Esto se cumple si $a = b = \frac{1}{n}$. De esta manera se obtiene:

\begin{eqnarray*}
	\sum_{k=1}^{n-1} \frac{1}{k(n - k)} &=& \frac{1}{n}\sum_{k=1}^{n-1} \frac{1}{k} + \frac{1}{n}\sum_{k=1}^{n-1} \frac{1}{n - k} \\
	                                    &=& \frac{1}{n}\mathcal{H}_{n-1} + \frac{1}{n}\left( \frac{1}{n - 1} + \frac{1}{n - 2} + \frac{1}{n - 3} + \ldots + \frac{1}{2} + 1 \right) \\
	                                    &=& \frac{1}{n}\mathcal{H}_{n-1} + \frac{1}{n}\sum_{h=1}^{n-1} \frac{1}{h} \\
	                                    &=& \frac{1}{n}\mathcal{H}_{n-1} + \frac{1}{n}\mathcal{H}_{n-1} \\
	                                    &=& \frac{2}{n}\mathcal{H}_{n-1}
\end{eqnarray*}
\smallskip

\begin{equation}
\boxed
{
\sum_{k=1}^{n-1} \frac{1}{k(n - k)} = \frac{2}{n}\mathcal{H}_{n-1}
}
\end{equation}
\smallskip

\noindent b) La expresi�n $\sum_{0<k<n} \frac{1}{k(n - k)}$ se puede ver como la convoluci�n de $\Braket{\mathcal{G}_k} = \Braket{1, \frac{1}{2},\frac{1}{3},\ldots}$ consigo mismo

\begin{equation*}
\sum_{k=1}^{n-1} \mathcal{G}_k\,\mathcal{G}_{n-k} = \left[z^n\right]\mathcal{G}(z)^2
\end{equation*}
\smallskip

\noindent donde $\mathcal{G}(z)$ es la funci�n generatriz de $\Braket{\mathcal{G}_k}$. La forma cerrada de esta funci�n generatriz se encuentra en la tabla 335 de \cite[pp. 335]{ConcreteMath}:

\begin{equation*}
	\mathcal{G}(z) = \ln\left(\frac{1}{1-z}\right)
\end{equation*}
\smallskip

\noindent Ahora, utilizando la ecuaci�n 7.50 de \cite[pp. 351]{ConcreteMath} se obtiene

\begin{eqnarray*}
	\sum_{k=1}^{n-1} \frac{1}{k(n - k)} &=& \left[z^n\right]\mathcal{G}(z)^2 \\
	                                    &=& \left[z^n\right]\left\{\ln\left(\frac{1}{1-z}\right)\right\}^2 \\
	                                    &=& \frac{2}{n!}\left[\begin{array}{c}  n \\ 2 \end{array}\right]
\end{eqnarray*}
\smallskip

\noindent Finalmente, para obtener la soluci�n en la misma forma que en la parte (a), solo se debe remplazar la identidad

\begin{equation*}
	\left[\begin{array}{c}  n \\ 2 \end{array}\right] = (n-1)! \, \mathcal{H}_{n-1} \quad\quad\quad \text{\cite[eq. 6.58]{ConcreteMath}}
\end{equation*}
\smallskip

\noindent en la expresi�n anterior.

\begin{equation}
\boxed
{
\sum_{k=1}^{n-1} \frac{1}{k(n - k)} = \frac{2}{n}\mathcal{H}_{n-1}
}
\end{equation}
\bigskip

\noindent {\bf Problema \#3:} Proporcione una formula asintotica para el coeficiente trinomial ``medio'' $\binom{3n}{n,n,n}$ con un error relativo $O(n^{-3})$.\\

\noindent {\bf Soluci�n}\\

Al remplazar la aproximaci�n de Stirling \cite[eq. 9.91]{ConcreteMath}

\begin{equation*}
	\ln\left(z!\right) = \frac{1}{2}\ln\left(2\pi\right) + \left(z+\frac{1}{2}\right)\ln\left(z\right) - z + \frac{1}{12}z^{-1} + O(z^{-3})
\end{equation*}
\smallskip

\noindent en el logaritmo neperaino de la formula del coeficiente trinomial

\begin{eqnarray*}
	\ln\binom{3n}{n,n,n} &=& \ln\left[\frac{(3n)!}{(n!)^3}\right] \\
	                     &=& \ln\left[(3n)!\right] - 3\ln\left[n!\right]
\end{eqnarray*}
\smallskip

\noindent se obtiene

\begin{eqnarray*}
	\ln\binom{3n}{n,n,n} &=& \;\;\, \frac{1}{2}\ln\left(2\pi\right) + \left(3n+\frac{1}{2}\right)\ln\left(3n\right) - 3n + \frac{1}{12}(3n)^{-1} \\
	                     & & - \frac{3}{2}\ln\left(2\pi\right) - 3\left(n+\frac{1}{2}\right)\ln\left(n\right) + 3n - \frac{3}{12}n^{-1} + O(n^{-3}) \\
	                     &=& - \ln\left(2\pi\right) + \left(3n+\frac{1}{2}\right)\ln\left(3\right) + \left(3n+\frac{1}{2}\right)\ln\left(n\right) - \left(3n+\frac{3}{2}\right)\ln\left(n\right) \\
	                     & & + \left(\frac{1}{36} - \frac{1}{4}\right)n^{-1} + O(n^{-3}) \\
 	                     &=& - \ln\left(2\pi\right) + \left(3n+\frac{1}{2}\right)\ln\left(3\right) - \ln\left(n\right) + \left(\frac{1}{36} - \frac{1}{4}\right)n^{-1} + O(n^{-3}) \\
 	                     &=& \;\;\, \ln\left(\frac{3^{3n+\frac{1}{2}}}{2\pi\,n}\right) - \frac{2}{9}n^{-1} + O(n^{-3})
\end{eqnarray*}
\smallskip

\noindent Luego, al hacer uso de la funci�n exponencial en ambos lados de la ecuaci�n 

\begin{eqnarray*}
	\binom{3n}{n,n,n} &=& \frac{3^{3n+\frac{1}{2}}}{2\pi\,n}\exp\left(-\frac{2}{9}n^{-1} + O(n^{-3})\right)
\end{eqnarray*}
\smallskip

\noindent y utilizar la aproximaci�n asintotica de la funci�n exponencial \cite[eq. 9.32]{ConcreteMath} 

\begin{eqnarray*}
	\exp(z) &=&  1 + z + \frac{1}{2}z^2 + O(z^{3})
\end{eqnarray*}
\smallskip

\noindent se obtiene finalmente

\begin{equation}
\boxed
{
	\binom{3n}{n,n,n} = \frac{3^{3n+\frac{1}{2}}}{2\pi\,n}\left[1 - \frac{2}{9}n^{-1} + \frac{2}{81}n^{-2} + O(n^{-3})\right]
}
\end{equation}
\smallskip

\noindent {\bf Problema \#2:} Sea $U(n,m)$ el n�mero de maneras distintas de sentar $n$ alumnos en una fila de $m$ pupitres dejando al menos un pupitre vac�o entre alumnos.

\begin{enumerate}
	\item [a)] Determine la funci�n generatriz de la secuencia $U(n,0)$, $U(n,1)$, $U(n,2)$, $\ldots$
	\item [b)] Halle una forma cerrada para $U(n,m)$.
\end{enumerate}
\smallskip

\noindent {\bf Soluci�n}\\

\noindent a) Primero se debe construir la relaci�n de recurrencia para la funci�n $U(n,m)$. Es sencillo notar que $U(n,m)$ es el numero de maneras de sentar $n$ alumnos en una fila $m-1$ [$U(n, m-1)$], m�s el numero de de maneras de sentarlos dejando un alumno sentado en el ultimo pupitre [$K(n,m)$].

\begin{equation*}
	U(n,m) = U(n, m-1) + K(n,m)
\end{equation*}
\smallskip

\noindent Al final un alumno en el ultimo pupitre, se deben acomodar $n-1$ estudiantes en $m-1$ pupitres. Sin embargo, debido a la restricci�n, el pen�ltimo pupitre debe estar vac�o, y por lo tanto realmente se deben acomodar los $n-1$ estudiantes restantes en $m-2$ pupitres.

\begin{equation*}
	K(n,m) = nU(n-1, m-2)
\end{equation*}
\smallskip

\noindent Considerando los casos base se tiene que la relaci�n de recurrencia esta dada por,

\begin{equation*}
	U(n,m) = \begin{cases}
	  0 & \text{si } n = 0 \vee m < 2n,\\
	  m & \text{si } n = 1,\\ 
	  U(n, m-1) + n\, U(n-1, m-2) & \text{en el resto de los casos}
	\end{cases}
\end{equation*}
\smallskip

\noindent Luego, al remplazar la relaci�n de recurrencia en la funci�n generatriz se obtiene

\begin{eqnarray*}
  G(z;n) & = & \sum_{m=0}^{\infty} U(n,m)z^m \\
         & = & 0 + nz + \sum_{m=2}^{\infty} U(n,m)z^m \\
         & = & nz + \sum_{m=2}^{\infty} U(n,m-1)z^m + n\sum_{m=2}^{\infty} U(n-1,m-2)z^m \\
         & = & nz + \left(\sum_{m=1}^{\infty} U(n,m-1)z^m - U(n,0)z \right) + n\sum_{m=2}^{\infty} U(n-1,m-2)z^m \\
         & = & nz + \sum_{m=1}^{\infty} U(n,m-1)z^m + n\sum_{m=2}^{\infty} U(n-1,m-2)z^m \\
         & = & nz + \sum_{k=0}^{\infty} U(n,k)z^{k+1} + n\sum_{k=0}^{\infty} U(n-1,k)z^{k+2} \\
         & = & nz + zG(z;n) + nz^{2}G(z;n-1) \\
         & = & \frac{nz^{2}}{1 - z}G(z;n-1) + \frac{nz}{1 - z} \\
         & = & \frac{nz}{1 - z}\left[zG(z;n-1) + 1\right] 
\end{eqnarray*}
\bigskip

\begin{thebibliography}{99}

\bibitem{ConcreteMath}
R.L. Graham, D.E. Knuth, and O. Patashnik, \emph{Concrete Mathematics}, Addison-Wesley, Reading, MA, 1989.
\end{thebibliography}

\end{document} 
