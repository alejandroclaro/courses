% ********************************************** %
% *                                            * %
% * Autor : Claro Mosqueda Alejandro           * %
% *                                            * %
% * e-mail: alejandro.claro@gmail.com          * %
% *                                            * %
% * Teoria de algoritmos, USB                  * %
% *                                            * %
% * Tarea No  2                                * %
% *                                            * %
% ********************************************** %

\documentclass[letterpaper,11pt]{article}

\usepackage[spanish]{babel}
\usepackage[latin1]{inputenc}
\usepackage{pifont}
\usepackage{mathrsfs}
\usepackage{stmaryrd, manfnt,marvosym,yhmath}
\usepackage{amsmath, amsthm, amssymb}
%\usepackage{amsfonts}
%\usepackage{graphicx}
%\usepackage{concmath}

\headheight 0.2truecm \headsep 0.5truecm \topmargin -0.54truecm
\topskip 0pt \textheight 22.49truecm \footskip 0.75truecm
\oddsidemargin 0.46truecm \evensidemargin 0.46truecm
\marginparwidth 0pt \marginparsep 0pt \textwidth 16.59truecm
\parindent 2em
\fboxsep 1ex \fboxrule 0.15ex \setlength{\jot}{0.2truecm}
\flushbottom

% Use $$\Braket{\phi | {\partial^2}\over {\partial t^2} | \psi}$$

\def\bra#1{\mathinner{\langle{#1}|}}
\def\ket#1{\mathinner{|{#1}\rangle}}
\def\braket#1{\mathinner{\langle{#1}\rangle}}
\def\Bra#1{\left<#1\right|}
\def\Ket#1{\left|#1\right>}
{\catcode`\|=\active 
  \gdef\Braket#1{\left<\mathcode`\|"8000\let|\bravert {#1}\right>}}
\def\bravert{\egroup\,\vrule\,\bgroup}

\newcommand{\pderiv}[2]{\frac{\partial #1}{\partial #2}}      % derivada parcial
\newcommand{\deriv}[2]{\frac{\mathrm{d} #1}{\mathrm{d} #2}}      % derivada parcial
\newcommand{\comm}[2]{\left[\,#1\;,\;#2\,\right]}                 % [ , ]
%\newcommand{\qed}{\hfill \ensuremath{\Box}}

%\newtheorem{theorem}{Theorem}[section]
%\newtheorem{lemma}[theorem]{Lemma}
%\newtheorem{proposition}[theorem]{Proposition}
%\newtheorem{corollary}[theorem]{Corollary}

%\newenvironment{proof}[1][Proof]{\begin{trivlist}
%\item[\hskip \labelsep {\bfseries #1}]}{\end{trivlist}}
%\newenvironment{definition}[1][Definition]{\begin{trivlist}
%\item[\hskip \labelsep {\bfseries #1}]}{\end{trivlist}}
%\newenvironment{example}[1][Example]{\begin{trivlist}
%\item[\hskip \labelsep {\bfseries #1}]}{\end{trivlist}}
%\newenvironment{remark}[1][Remark]{\begin{trivlist}
%\item[\hskip \labelsep {\bfseries #1}]}{\end{trivlist}}

%%%%%%%%%%%%%%% Comienzo del documento %%%%%%%%%%%%%%

\begin{document}

  \setlength{\unitlength}{1truecm}\thicklines%

  \begin{flushleft}
    \bf {\small UNIVERSIDAD SIM�N BOLIVAR\hfill \today} \\
    \bf {\small CI-7521 - Matem�ticas para la computaci�n\hfill Lic. Alejandro Claro Mosqueda} \\
  \end{flushleft}

  \bigskip

	\begin{center}
	\noindent \Large \bf Tarea \#2
	\end{center}

  \bigskip

\noindent {\bf Problema \#1:} Calcular la suma ${\sum\limits_{1\leq k \leq n} \lfloor\log_2k\rfloor}$, donde $a$ es la parte entera por debajo de $a$. \\

\noindent {\bf Soluci�n}\\
\smallskip

\noindent Siguiendo el procedimiento explicado en la secci�n \textbf{3.5}\cite{ConcreteMath} se obtiene que

\begin{eqnarray*}
	 \sum_{1\leq k \leq n} \lfloor\log_2k\rfloor &=& \sum_{k,m \geq 0} m\left[m = \lfloor\log_2k\rfloor\right]\left[ 1 \leq k \leq n\right]\\
&=& \sum_{k,m \geq 0} m\left[ 1 \leq k \leq n \right]\left[ m   \leq \log_2k < m + 1    \right]\\
&=& \sum_{k,m \geq 0} m\left[ 1 \leq k \leq n \right]\left[ 2^m \leq k       < 2^{m + 1}\right] \\
\end{eqnarray*}

\noindent Para cada entero positivo $k$ tal que $2^m \leq k < 2^{m + 1}$, donde $m = \lfloor\log_2k\rfloor$, se tiene que hay $2^{m+1} - 2^m = 2^m$ t�rminos iguales a $m$ siempre y cuando $m < \lfloor\log_2n\rfloor$. Tambi�n es f�cil reconocer que hay $n - 2^{\lfloor\log_2n\rfloor} + 1$ t�rminos en la sumas que son iguales a $\lfloor\log_2n\rfloor$. Por cual se obtiene que

\begin{eqnarray*}
	 \sum_{1\leq k \leq n} \lfloor\log_2k\rfloor &=& \sum_{m \geq 0} m\,2^m \left[ m < \lfloor\log_2n\rfloor \right] + \lfloor\log_2n\rfloor \left(n - 2^{\lfloor\log_2n\rfloor} + 1\right)\\
&=& \sum_{m = 0}^{\lfloor\log_2n\rfloor - 1} m\,2^m + \lfloor\log_2n\rfloor \left(n - 2^{\lfloor\log_2n\rfloor} + 1\right)\\	&=& \sum_{x = 0}^{\lfloor\log_2n\rfloor} x\,2^x\delta x + \lfloor\log_2n\rfloor \left(n - 2^{\lfloor\log_2n\rfloor} + 1\right)\\ 
\end{eqnarray*}

\noindent Encontrar una forma cerrada para la suma $\sum_{0}^{\lfloor\log_2n\rfloor} x\,2^x\delta x$ es sencillo al realizar una suma por partes con $u(x) = x$ y $\Delta v = 2^x$,

\begin{eqnarray*}
	 \sum_{x = 0}^{\lfloor\log_2n\rfloor} x\,2^x\delta x &=& x\,2^x\Bigg\vert_{x=0}^{\lfloor\log_2n\rfloor} - \sum_{x = 0}^{\lfloor\log_2n\rfloor}2^{x+1} \\
&=& x\,2^x - 2^{x+1} \Bigg\vert_{x=0}^{\lfloor\log_2n\rfloor} \\	 
&=& {\lfloor\log_2n\rfloor}\,2^{\lfloor\log_2n\rfloor} - 2^{{\lfloor\log_2n\rfloor}+1} + 2
\end{eqnarray*}

\noindent Remplazando esta suma se obtiene finalmente que

\begin{eqnarray*}
	 \sum_{1\leq k \leq n} \lfloor\log_2k\rfloor &=& {\lfloor\log_2n\rfloor}\,2^{\lfloor\log_2n\rfloor} - 2^{{\lfloor\log_2n\rfloor}+1} + 2 + \lfloor\log_2n\rfloor \left(n - 2^{\lfloor\log_2n\rfloor} + 1\right)
\end{eqnarray*}

\begin{equation*}
	 \boxed{\sum_{k = 1}^{n} \lfloor\log_2k\rfloor = \lfloor\log_2n\rfloor \left(n + 1\right) - 2\left( 2^{{\lfloor\log_2n\rfloor}} - 1\right)}
\end{equation*}
\bigskip

\noindent {\bf Problema \#2:} Eval�e $\mancube_n = \sum_{k=1}^n k^3$ utilizando el m�todo 5 del libro \textbf{Concrete Mathematics}\cite{ConcreteMath} como sigue: primero escriba $\mancube_n + \Box_n = 2\sum_{1 \leq j \leq k \leq n} jk$; luego aplique (\textbf{2.33})\cite{ConcreteMath}. \\

\noindent {\bf Soluci�n}\\

\noindent Primero observese que,

\begin{eqnarray*}
	 \mancube_n + \Box_n &=& \sum_{k=1}^n k^3 + \sum_{k=1}^n k^2 \\
	                     &=& \sum_{k=1}^n \left\{ k^3 + k^2 \right\} \\
	                     &=& \sum_{k=1}^n \left\{ k^2 (k + 1) \right\} \\
	                     &=& \sum_{k=1}^n \left\{ 2\,k \left[\frac{k(k + 1)}{2}\right] \right\} \\
	                     &=& 2\sum_{k=1}^n \left\{ k \sum_{j=1}^{k}j \right\} \\
	                     &=& 2\sum_{1 \leq j \leq k \leq n} jk
\end{eqnarray*}
\smallskip

\noindent Aprovechando este resultado y (\textbf{2.33})\cite{ConcreteMath}

\begin{equation*}
	S_{\triangledown} = \sum_{1 \leq j \leq k \leq n} a_j\,a_k = \frac{1}{2}\left[ \left(\sum_{k=1}^n a_k\right)^2 + \sum_{k=1}^n a_k^2 \right]
\end{equation*}
\smallskip

\noindent se obtiene, al sustituir $a_j=j$ y $a_k=k$, que

\begin{eqnarray*}
	 \frac{\mancube_n + \Box_n}{2} &=& \sum_{1 \leq j \leq k \leq n} jk \\
	                               &=& \frac{1}{2}\left[ \left(\sum_{k=1}^n k\right)^2 + \sum_{k=1}^n k^2 \right]\\
                                 &=& \frac{1}{2}\left\{ \left[\frac{n(n+1)}{2} \right]^2 + \Box_n \right\}\\
                                 &&\\
                     \mancube_n  &=& \left[\frac{n(n+1)}{2} \right]^2
\end{eqnarray*}
\smallskip

\begin{equation*}
   \boxed{\mancube_n = \frac{\left[n(n+1)\right]^2}{4}}
\end{equation*}
\bigskip

\noindent {\bf Problema \#3:} Eval�e la suma $\sum\limits_{k=1}^{n}\frac{(2k+1)}{k(k+1)}$ usando sumas por partes.\\

\noindent {\bf Soluci�n}\\

\noindent Observese que,

\begin{eqnarray*}
	 \sum_{k=1}^{n}\frac{(2k+1)}{k(k+1)} &=& \sum_{j=0}^{n-1}\frac{(2j+3)}{(j+1)(j+2)} \qquad\qquad (j = k -1)\\
	                                     &=& \sum_{j=0}^{n-1}(2j+3)j^{\underline{-2}} \\
	                                     &=& \sum_{j=0}^{n}(2x^{\underline{1}}+3)x^{\underline{-2}}\delta x
\end{eqnarray*}
\smallskip

\noindent Ahora, al sumar por partes con,

\begin{eqnarray*}
  u(x)     &=& 2x^{\underline{1}}+3 \\
  \Delta u &=& 2 \\
  \Delta v &=& x^{\underline{-2}}   \\
  v(x)     &=& - x^{\underline{-1}} 
\end{eqnarray*}
\smallskip

\noindent se obtiene que

\begin{eqnarray*}
	 \sum_{j=0}^{n}(2x^{\underline{1}}+3)x^{\underline{-2}}\delta x &=& -(2x^{\underline{1}}+3)x^{\underline{-1}}\Bigg\vert_{x=0}^n + 2\sum_{j=0}^{n}(x+1)^{\underline{-1}}\delta x\\
	 &=& 2H_{x+1}-(2x^{\underline{1}}+3)x^{\underline{-1}}\Bigg\vert_{x=0}^n\\
	 &=& 2H_{n+1} - \dfrac{2n+3}{n+1} - \left[ 2H_1 - 3\right]\\
	 &=& 2H_{n+1} - \frac{n + 2}{n+1}\\
	 &=& 2H_{n} + \frac{2}{n+1} - \frac{n + 2}{n+1}	  \\
	 &=& 2H_{n} - \frac{n}{n+1}	 
\end{eqnarray*}
\smallskip

\noindent donde $H_{n}$ es el $(n)$-�simo n�mero arm�nico.

\begin{equation*}
   \boxed{\sum_{k=1}^{n}\frac{(2k+1)}{k(k+1)} = 2H_{n} - \frac{n}{n+1}}
\end{equation*}
\bigskip

\noindent {\bf Problema \#4:} Determine una forma cerrada para la suma $\sum\limits_{0 \leq i \leq n}\frac{i^2\,4^{i-1}}{(i+1)(i+2)}$.\\

\noindent {\bf Soluci�n}\\

\begin{eqnarray*}
	 \sum_{0 \leq i \leq n}\frac{i^2\,4^{i-1}}{(i+1)(i+2)} &=& \frac{1}{4}\sum_{0 \leq i \leq n}\frac{i^2\,4^{i}}{(i+1)(i+2)} \\
	 &=& \frac{1}{4}\sum_{0 \leq i \leq n}4^{i}\left[\frac{i^2 + 4i + 4 - 4i - 4}{(i+1)(i+2)}\right]\\
	 &=& \frac{1}{4}\sum_{0 \leq i \leq n}4^{i}\left[\frac{(i+2)^2 - 4i - 4}{(i+1)(i+2)}\right]\\
	 &=& \frac{1}{4}\sum_{0 \leq i \leq n}4^{i}\left[\frac{i+2}{i+1} - \frac{4i + 4}{(i+1)(i+2)}\right]\\
	 &=& \frac{1}{4}\sum_{0 \leq i \leq n}4^{i}\left[\frac{i+2}{i+1} - \frac{4}{i+2}\right]\\
	 &=& \frac{1}{4}\left[\sum_{0 \leq i \leq n}4^{i}\left(\frac{i+2}{i+1}\right) - \sum_{1 \leq k \leq n+1}4^{k}\left(\frac{1}{k+1}\right)\right] \qquad \qquad (k = i + 1)\\	 
	 &=& \frac{1}{4}\left[ 2 + \sum_{1 \leq i \leq n}4^{i}\left(\frac{i+2}{i+1}\right) - \sum_{1 \leq k \leq n}4^{k}\left(\frac{1}{k+1}\right) - 4^{n+1}\left(\frac{1}{n+2}\right)\right]\\
\end{eqnarray*}
\smallskip

\begin{eqnarray*}
	 \sum_{0 \leq i \leq n}\frac{i^2\,4^{i-1}}{(i+1)(i+2)} &=& \frac{1}{4}\left[ 2 + \sum_{1 \leq i \leq n}4^{i} - 4^{n+1}\left(\frac{1}{n+2}\right)\right]\\	 
	 &=& \frac{1}{4}\left[ 2 + \sum_{0 \leq i \leq n}4^{i} - 1 - 4^{n+1}\left(\frac{1}{n+2}\right)\right]\\	 
	 &=& \frac{1}{4}\left[ 1 + \left( \frac{1 - 4^{n+1}}{1 - 4} \right) - 4^{n+1}\left(\frac{1}{n+2}\right)\right]\\	 
	 &=& \frac{1}{12}\left[ 2 + 4^{n+1} - 3\left(\frac{4^{n+1}}{n+2}\right)\right]\\	 
	 &=& \frac{1}{3}\left[ \frac{1}{2} + 4^{n} - 3\left(\frac{4^{n}}{n+2}\right)\right]\\	 	 
	 &=& \frac{1}{6} - \frac{4^{n}}{3}\left(\frac{n-1}{n+2}\right)\\	 
\end{eqnarray*}
\smallskip

\begin{equation*}
   \boxed{\sum_{i=0}^{n}\frac{i^2\,4^{i-1}}{(i+1)(i+2)} = \frac{1}{6} - \frac{4^{n}}{3}\left(\frac{n-1}{n+2}\right)}
\end{equation*}
\bigskip

\begin{thebibliography}{99}

\bibitem{ConcreteMath}
R.L. Graham, D.E. Knuth, and O. Patashnik, \emph{Concrete Mathematics}, Addison-Wesley, Reading, MA, 1989.
\end{thebibliography}

\end{document} 
